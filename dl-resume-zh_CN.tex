% !TEX TS-program = xelatex
% !TEX encoding = UTF-8 Unicode
% !Mode:: "TeX:UTF-8"

\documentclass{resume}
\usepackage{zh_CN-Adobefonts_external} % Simplified Chinese Support using external fonts (./fonts/zh_CN-Adobe/)
%\usepackage{zh_CN-Adobefonts_internal} % Simplified Chinese Support using system fonts
\usepackage{linespacing_fix} % disable extra space before next section
\usepackage{cite}

\begin{document}
\pagenumbering{gobble} % suppress displaying page number

\name{张鹏程}

\basicInfo{
  \email{holmesconan@gmail.com} \textperiodcentered\ 
  \phone{(+86) 189-1005-3803} \textperiodcentered\ 
  \linkedin[LinkedIn]{https://www.linkedin.com/in/zhang-pengcheng-3a626461/}}
 
\section{\faGraduationCap\  教育背景}
\datedsubsection{\textbf{河北大学}, 河北, 保定}{2009 -- 2012}
\textit{硕士研究生}\ 理论物理
\datedsubsection{\textbf{河北大学}, 河北, 保定}{2004 -- 2008}
\textit{学士}\ 应用物理

\section{\faUsers\ 项目经历}
\datedsubsection{\textbf{FFTNet-CUDA}}{2018年8月 -- 至今}
\role{CUDA/cuDNN}{实验项目}
\begin{onehalfspacing}
基于 CUDA/cuDNN 和 OpenCV 的深度学习框架,目标是提供内存优化和方便的 Python 转 C++ 开发,
以提供更好的产生部属性能。
\begin{itemize}
  \item 使用 C++ RAII  机制封装了 cudnnHandle\_t、cudnn*Descriptor\_t,实现资源生命周期的自动管理
  \item 实现了一套 Exception 处理机制,方便进行调试
  \item 实现了一个多维数组的简单封装,完成数据的填充和内存管理
  \item 使用了「测试驱动开发」的方式,极大的减少了开发的时间
\end{itemize}
\end{onehalfspacing}

\datedsubsection{\textbf{FFTNet-CUDA}}{2018年8月1日 -- 2018年8月11日}
\role{CUDA/cuDNN}{实验项目}
\begin{onehalfspacing}
基于 CUDA/cuDNN 完成的 FFTNet Forward pass 开发。
\begin{itemize}
  \item 使用 C++ RAII  机制封装了 cudnnHandle\_t、cudnn*Descriptor\_t,实现资源生命周期的自动管理
  \item 实现了一套 Exception 处理机制,方便进行调试
  \item 实现了一个多维数组的简单封装,完成数据的填充和内存管理
  \item 使用了「测试驱动开发」的方式,极大的减少了开发的时间
\end{itemize}
\end{onehalfspacing}

\datedsubsection{\textbf{vDNN}}{2018年6月 -- 2018年7月}
\role{Tensorflow}{个人项目}
\begin{onehalfspacing}
参考 vDNN/moDNN 和阿里在 NIPS 2017 Workshop 上的一篇论文,实现了一个在小内存下训练大网络的技术原型。
\begin{itemize}
  \item 通过使用 Tensorflow 的内部 API,切断了 forward pass 中的 tensor 与 gradients 的连接
  \item 使用 tf.identity 把中间结果转移到 host 内存
  \item 结果:成功将部分 tensor 转移到了 host,但并没有明显提高 batch 的大小
\end{itemize}
\end{onehalfspacing}

\datedsubsection{\textbf{K 线形态识别}}{2018年3月 -- 2018年7月}
\role{Tensorflow}{个人项目}
\begin{onehalfspacing}
使用图像分类技术,实现对 K 线形态的识别,进而为交易提供参考信息。
\begin{itemize}
  \item 基于 VGG-16/ResNet/MobileNet/MobileNet v2/ShuffleNet 等 2016 年以后发表的模型
  \item 使用 2016 年至今的现实数据生成 K 线图片
  \item 结果:准确率达到了 65 \% 以上
\end{itemize}
\end{onehalfspacing}

\datedsubsection{\textbf{StockGame}}{2018年7月 -- 2018年7月}
\role{PyTorch}{个人项目}
\begin{onehalfspacing}
参考 OpenAI/Gym Game 的规范,构建的一个股票交易游戏。实现训练一个基于强化学习的股票交易员。
\begin{itemize}
  \item 基于 PyTorch 的 Actor-Critic 模型
  \item 创建了一个股票交易的模拟环境,包括:交易策略、游戏状态空间和回报函数
  \item 结果:在单支股票上的平均收益可以达到 20 \% 以上
\end{itemize}
\end{onehalfspacing}

\datedsubsection{\textbf{广州思拓信息科技有限公司}, 广州}{2015年03月 -- 至今}
\role{全栈开发}{完全自主开发}
\begin{onehalfspacing}
客户关系管理系统(2016年11月至今)
\begin{itemize}
  \item 基于 PHP/Laravel
  \item 基于 Bootstrap/Gentelella
  \item 个人完成九成的代码开发
\end{itemize}
\end{onehalfspacing}
\begin{onehalfspacing}
广州入户评估微信小程序
\begin{itemize}
  \item 基于 WeUI
  \item 原生环境开发
  \item 表单型小程序
  \item 与公司客户关系管理系统整合
\end{itemize}
\end{onehalfspacing}

\datedsubsection{\textbf{北京土星创游}, 北京}{2013年3月 -- 2014年7月}
\role{C++, Cocos2d-x}{主管:崔嵬}
\begin{onehalfspacing}
「乱世将神OL」手机游戏前端开发
\begin{itemize}
  \item 多线程更新模块
  \item 基于 C++ Template 的反序列化库
  \item 多个游戏模块的开发
\end{itemize}
\end{onehalfspacing}

\datedsubsection{\textbf{中国科学院高能物理研究所} 北京}{2012年7月 -- 2013年3月}
\role{C++}{主管: 宋黎明}
硬 X 射线调制望远镜(卫星)地面数据系统开发
\begin{itemize}
  \item 数据标准定义(参与)
  \item 低能量段数据处理
  \item 地面数据系统处理架构设计
\end{itemize}

\section{\faCogs\ IT 技能和评级}
% increase linespacing [parsep=0.5ex]
\begin{itemize}[parsep=0.5ex]
  \item 编程语言: C/C++ -> Competent
  \item 编程语言: Python -> Competent
  \item 编程语言: PHP -> Competent
  \item 编程语言: JavaScript -> Advanced Beginner
  \item 编程语言: CSS/HTML -> Advanced beginner
  \item 框架: Tensorflow -> Advanced beginner
  \item 框架: PyTorch -> Advanced beginner
  \item 框架: CUDA/cuDNN -> Advanced beginner
  \item 平台: Linux
\end{itemize}

\section{\faInfo\ 其他}
% increase linespacing [parsep=0.5ex]
\begin{itemize}[parsep=0.5ex]
  \item 技术博客: http://www.cnblogs.com/holmescn/
  \item 个人主页: http://www.holmesconan.me
  \item GitHub: https://github.com/holmescn
  \item 语言: 英语 - 熟练(足够阅读英语文档)
\end{itemize}

\end{document}
